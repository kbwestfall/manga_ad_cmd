%-------------------------------------------------------------------------------
% Kyle Westfall
% westfall@ucolick.org
% UCO/Lick Observatory
% University of California, Santa Cruz
% 1156 High St.
% Santa Cruz, CA 95064
% USA
%
% VERSION:
%       05 Feb 2017: (KBW) v01: stab in the dark
%       01 Aug 2017: (KBW) edits and summary of todo tasks
%
% SUBMITTED:
%
%-------------------------------------------------------------------------------

%\input{../dm_nomenclature/dmnom_rv4.tex}

\documentclass[apj,iop,revtex4,numberedappendix]{emulateapj}
\RequirePackage{calc}
\usepackage{natbib}
\usepackage{url}
\usepackage{amsmath}
\usepackage{mathrsfs}
\usepackage{graphicx}
\usepackage{color}
\bibliographystyle{apj}

\slugcomment{Draft: 22 Jan 2018}

% Some fancy commenting
\definecolor{todo}{RGB}{200,0,0}
\newcommand{\comment}[2][todo]{{\color{#1}[[{\bf #2}]]}}

% Some shorthands
\newcommand{\kms}{{km$~\!$s$^{-1}$}}
\newcommand{\nsaz}{$z_{\rm NSA}$}
\newcommand{\halpha}{H$\alpha$}

\shortauthors{Westfall et al.}
\shorttitle{Asymmetric Drift across the Blue Cloud}

\begin{document}

\title{ Trends in asymmetric drift across the blue cloud }

\author{ Kyle B. Westfall\altaffilmark{1}, Matthew A.
Bershady\altaffilmark{2}, Kevin Bundy\altaffilmark{1}, et al. }

\altaffiltext{1}{University of California Observatories, University of
California Santa Cruz, 1156 High St., Santa Cruz, CA 95064, USA}

\altaffiltext{2}{Department of Astronomy, University of Wisconsin--Madison, 475
N. Charter St., Madison, WI 53706, USA}

\email{westfall@ucolick.org}

% Project collaborators: Matthew Bershady, Anne-Marie Weijmans, Karen
% Masters, Michael Merrifield, Jo Bovy, Kevin Bundy, Niv Drory, Thomas
% Martinsson, Joel Brownstein, Daniel Thomas, Dmitry Bizyaev, David Law,
% Andres Meza, Aaron A. Dutton

\begin{abstract}

Asymmetric drift (AD) is the lag of the mean rotation velocity of the
stellar disk behind the circular speed defined by the total
gravitational potential.  Although it is often considered a nuisance
correction one must apply in circular-speed calculations, the direct
connection between AD and the stellar phase-space distribution function
makes it an interesting dynamical properties of galaxies in and of
itself.  The SDSS-IV/MaNGA survey provides more than an order of
magnitude increase in any galaxy sample size useful for AD measurements.
In this pilot study, we measure AD --- or more precisely the
differential tangential lag between the stellar component and
\halpha-emitting gas --- in a set of galaxies with \halpha\ and stellar
velocity fields that are well-fit by simple disk models.  We describe
our fitting approach and the selection of our subsample in detail.  Our
final sample of $\sim$XXX galaxies shows a clear correlation between
absolute $i$-band magnitude and AD measured at effectively all radii.
\comment{The rest neesds to be updated.} Removing the primary dependence
on the rotation speed, we find that the stellar disk rotation is 90\% of
the \halpha\ rotation speed, in the mean.  Our sample size allows us to
infer a weak, yet statistically significant, trend with galaxy $N-r$
color such that, in the mean, redder galaxies have moderately larger AD
relative to their rotation speeds.  Within the context of an albeit
strong set of assumptions, including a direct proportionality between AD
and stellar velocity dispersion as seen in the Milky Way, we argue that
our results suggest little variation in the mean disk mass-to-light
ratio as a function of absolute magnitude and only modest variations
with galaxy color \comment{last sentence TBD; is Figure 3 consistent
with SPS variations?}.

\end{abstract}

\keywords{ galaxies: kinematics and dynamics --- galaxies: spiral ---
galaxies: structure }

\section{ Motivation }
\label{sec:intro}

For ensembles of stars in a galaxy disk, \citet[][Section
4.4.3]{2008gady.book.....B} provide an intuitive description of
asymmetric drift (AD).  The combined effect of the radially decreasing
surface-density and velocity-dispersion profiles, typical of
axisymmetric systems like the Milky Way, leads to an asymmetric velocity
distribution function with a mean value less than the circular speed
defined by the gravitational potential.  The standard mathematical
representation of this is derived by taking the $v_R$ moment of the
collisionless Boltzmann equation to find the Jeans equation
\citep{Jeans1919} that directly relates the circular speed ($v_c$), the
mean stellar tangential speed ($\overline{v_\theta}$), and the stellar
velocity ellipsoid (SVE) as a function of radius in the plane of
symmetry:
%
\begin{equation}
%
v_c^2 - \overline{v_\theta}^2 = \sigma_R^2\left[
\frac{\sigma_\theta^2}{\sigma_R^2} -
\frac{R}{\rho\sigma_R^2}\frac{\partial(\rho\sigma_R^2)}{\partial R} -
1\right] - R\frac{\partial\overline{v_R v_z}}{\partial z},
%
\label{eq:adformal}
%
\end{equation}
%
where $R,\theta,z$ are the cylindrical coordinates, $\rho$ is the volume
density, and $\sigma$ is the velocity dispersion.  Along with the
standard assumptions of dynamical equilibrium and negligible radial and
vertical flows inherent to its derivation, equation \ref{eq:adformal} is
often simplified by assuming the right-most term --- describing the
covariance between the radial and vertical motions as a function of
perpendicular distance to the plane of symmetry --- is negligible
\citep[cf.][]{1991ApJ...368...79A}.

Asymmetric drift has been measured in numerous systems, perhaps most
notably in the Galaxy.  For example, \citet{1998MNRAS.298..387D} have
shown that populations of stars show a direct correlation between their
velocity dispersion and the degree to which their mean rotation speed
lags behind that of the Local Standard of Rest (LSR), much in line with
the expectation provided by equation \ref{eq:adformal} (see also
\comment{more recent RAVE and RAVE+Gaia references}).

Asymmetric drift measurements in extragalactic systems are common in the
literature \ref{examples}; however, it is most often cast as a nuisance
phenomenon that one must correct for when constructing circular-speed
curves \comment{refs}.  The ubiquity and phenomenology of AD as a
salient observable has not yet been studied for a statistically
significant population of galaxies.  Our aim is therefore to provide a
first look at the correlation between AD measurements and basic
broad-band photometric properties for a large sample of galaxies.

%%%%%%%%%%%%%%%%%%%%%%%%%%%%%%%%%%%%%%%%%%%%%%%%%%%%%%%%%%%%%%%%%%%%%%%%
\begin{figure*}
%
\begin{center}
%
\includegraphics[width=0.9\textwidth]{figs/cmd_flux.pdf}
%
\end{center}
%
\caption{
%
${\rm NUV} - r$ color and absolute $i$-band magnitude ($M_i$) from the
NASA-Sloan Atlas (NSA) for all galaxies observed during the first two
years of the MaNGA Survey (left), the subsample of galaxies {\em not}
selected to be kinematically regular (middle; see text), and the AD
sample used throughout the remainder of our analysis.  The color of the
data points in the top row represent the mean \halpha\ surface
brightness within the elliptical-Petrosian half-light radius according
to the colorbar to the right; the bottom row replaces the color by the
mean $g$-band surface brightness density over the same elliptical
aperture.
%
}
%
\label{fig:sample}
%
\end{figure*}
%%%%%%%%%%%%%%%%%%%%%%%%%%%%%%%%%%%%%%%%%%%%%%%%%%%%%%%%%%%%%%%%%%%%%%%%

In addition to a basic characterization of the phenomenon, our interest
in AD stems from its fundamental dynamical connection to the full
phase-space distribution function of a galaxy's stars, as empirically
demonstrated in the Milky Way.  Indeed, \citet{2011ApJ...742...18W} used
AD to constrain the axial ratios of the SVE given direct measurements of
the line-of-sight (LOS) stellar velocity dispersion, $\sigma$.
Alternatively, if one can leverage statistical constraints on the shape
of the SVE \citep[e.g.][]{2012MNRAS.423.2726G}, measurements of AD can
be used as a proxy for stellar $\sigma$ via equation \ref{eq:adformal}.
Such a use of AD-$\sigma$ relation is attractive in the
low-surface-brightness and low-velocity-dispersion regimes where direct
measurements are difficult and/or expensive.  We commit to this paradigm
by defining $\sigma_a^2 \equiv v_c^2 - \overline{v_\theta}^2$, following
from equation \ref{eq:adformal}, for use throughout this paper.

For disk galaxies, there is a decades-long industry of using cold-gas
tracers to construct circular-speed curves of galaxies \comment{refs}.
These data have provided the first concrete arguments for the presence
of massive dark-matter halos \comment{ref} based on mass-model
reconstruction \comment{refs} and have formed the basis for the most
robust Tully-Fisher relations compiled for the local Universe
\comment{refs}.  However, it is important to acknowledge that one cannot
directly measure $v_c$, effectively making $\sigma_a$ unobservable.
That is, every dynamical tracer has some non-zero dynamical pressure
such that it will lag behind the theoretically defined circular speed.
There are very clear examples of early-type galaxies that show
differential AD in their molecular (e.g., CO), atomic (e.g.,
\ion{H}{1}), and/or ionized (\halpha) tracers relative to a robust mass
model of $v_c$ \citep{2013MNRAS.429..534D}, due to the different
turbulent/thermal pressures intrinsic to these tracers.  However, these
signatures are much less apparent in disk galaxies.  For example,
\citet{2013A&A...557A.131M} show that \halpha\ and \ion{H}{1} rotation
curves are consistent for the DiskMass Survey within the limits of their
constraints on beam-smearing.  This suggests that any correction for the
lag behind the circular speed of either the \halpha- or
\ion{H}{1}-emitting gas should be small.  Indeed, theoretical
calculations \citet{2010ApJ...721..547D} suggest that gas-pressure
corrections to \ion{H}{1} rotation curves should be largely negligible
for galaxies with circular speeds larger than roughly 75 \kms.
Nevertheless, we emphasize here that our measurements of $\sigma_a$ may
be better termed as a {\em differential tangential lag} because they are
calculated as the quadrature difference between the {\em observed}
\halpha\ and stellar rotation curves in our galaxy sample (Section
\ref{sec:data}).  These measurements are perfectly valid in their own
right; however, their interpretation in the context of the theoretical
definition of AD must be done with care, as we discuss in Section
\ref{sec:discussion}.

We present the data used for our analysis in the following section.
Section \ref{sec:results} presents the strong correlation seen in out
galaxy sample between the absolute $i$-band magnitude and AD signal
measured at half of an effective radius (0.5 $R_{\rm eff}$).  We also
illustrate the weak color dependence in this relation.  Finally, we
summarize and discuss these results in Section \ref{sec:discussion}.
\comment{flesh out}

\section{Data}
\label{sec:data}

We use integral-field spectroscopy from the SDSS-IV/MaNGA (Mapping
Nearby Galaxies from APO) Survey to construct stellar and ionized-gas
velocity fields for 2715 unique galaxies, 39 of which have multiple
observations.  These data were obtained during the first two years of
normal survey operations, and the reduced datacubes are included in DR14
\citep{2017arXiv170709322A}.  \comment{additional references to
technical papers}.

The kinematic measurements are determined by a preliminary version of
the MaNGA data analysis pipeline (DAP; Westfall et al., in prep).  From
this preliminary version (2.0.2), we use the simple single-Gaussian fits
to the \halpha\ emission feature for the ionized-gas kinematics, and the
stellar kinematics are determined using {\tt pPXF}
\citep{2004PASP..116..138C, 2017MNRAS.466..798C} with the MILES stellar
template library \citep{2011A&A...532A..95F}.  We find the velocity
measurements to be statistically well behaved to a $g$-band
signal-to-noise ratio (SNR) of $\sim$1.\footnote{
%
This and other assessments of the fidelity of the stellar kinematics
provided by the DAP will be discussed in detail by Westfall et al., {\em
in prep}.}
%
Therefore, we perform our analysis using the stellar kinematics measured
for each MaNGA spaxel (each spaxel is $0\farcs5$ on a side and
subsamples the $\sim$$2\farcs5$ fiber beam).

% \comment{Mention hierarchical clustering?}

The MaNGA galaxy selection, see \citet{2017AJ....154...86W}, is a simple
cut in absolute $i$-band magnitude, $M_i$, from the NASA-Sloan Atlas
(NSA; \url{www.nsatlas.org}) and the SDSS single-fiber redshift, $z$.
The MaNGA galaxy sample provides a statistically complete representation
of the overall galaxy population with a provided set of corrections to a
volume-limited sample.  The NSA-based ($N-r, M_i$) color-magnitude
diagram (CMD) of the full sample of galaxies obtained through the second
year of survey operations are shown in the left-most panels of Figure
\ref{fig:sample}, with the points colored by the \halpha\ .

Using the photometry from the NSA, Figure \ref{fig:sample} shows the
color-magnitude diagram of all galaxies colored by their mean \halpha\
and $r$-band surface brightness.

However, our nominal approach to measuring the stellar and gas rotation curves will be limited forAs such,
measurements of stellar and/or \halpha\ rotation curves for some systems
will be difficult/impossible.  



\comment{Update this} Although not
excluded from our velocity-field analysis {\em a priori}, we expect to
obtain poor rotation-curve measurements for galaxies with low \halpha\
and stellar surface brightness.\footnote{
%
The uniformity of the MaNGA survey \citet{2015AJ....150...19L} is such
that the signal-to-noise (S/N) of the datacube spaxels is tightly
correlated with surface brightness.}

\comment{Extinction corrections for the CMD?}

\comment{Primary vs. Secondary sample?}

We model the geometric projection of the rotational plane of each galaxy
using the approach presented by \citep[][see also
\citealt{2011ApJ...742...18W}]{2003ApJ...599L..79A}.  In three
independent fitting iterations, the model fits are optimized for the
\halpha\ velocity field, the stellar velocity field, and simultaneously
for both data sets; for the latter, the geometry is forced to be the
same for the two dynamical tracers, but the parametrized rotation curves
are independent.  We use these velocity-field-fitting results to
objectively isolate a set of ``kinematically regular'' galaxies.
Briefly, galaxies in this sample must have: (i) successful
velocity-field fits for all three approaches, (ii) differences in the
measured \halpha\ and stellar systemic velocity of less than 20 \kms,
(iii) dynamical centers that are consistent to within a fiber diameter
of the morphological center \comment{check the details of this}, (iv)
\halpha\ and stellar velocity-field position angles that are consistent
to within $\pm$15$\arcdeg$, and (v) kinematic inclinations between
$15\arcdeg < i < 80\arcdeg$ that are both consistent between the
\halpha\ and stellar data and with respect to the photometric
ellipticity to within $\pm$20$\arcdeg$.  The constraint on the
inclination is by far the most stringent. \comment{give number of
galaxies cut by each criterion?}  Applying these constraints yields a
sample of 798 observations (for 790 unique galaxies) out of the 2764
observations analyzed.  Eleven galaxies with repeat observations satisfy
the selection criteria; however, only five of these show all
observations are consistently selected.  The remaining six show one or
more of the observations did not pass our constraints \comment{we should
understand why repeat observations are not consistent for the majority
of cases.  low S/N?}.  \comment{Do this or remove the sentence:}
Finally, we also visually inspected the broad-band imaging of these 790
galaxies and eliminated merging and highly extincted (highly inclined)
systems yielding a final sample of XXX galaxies.  We hereafter refer to
galaxies that satisfy our selection criteria as the ``AD sample''.

\comment{limitations of an infinitely thin disk fit?}

Figure \ref{fig:sample} shows the color-magnitude distribution for all
MaNGA galaxies, as well as the distributions of those galaxies included
and excluded from our AD sample.  As expected, galaxies with relatively
high \halpha\ and $r$-band surface brightness are preferentially
selected.  This results in an exclusion of much of the red sequence, as
well as the brightest and faintest galaxies in the blue cloud.  Although
the conclusions we reach based on our AD sample are astrophysically
meaningful, it is important to appreciate that the galaxies in our AD
sample are a biased representation of the overall galaxy population.

\comment{Show histogram of kinematically regular sample against the
volume-corrected distribution of the MaNGA parent sample in Mi and N-r,
and discuss?}

Our velocity-fitting method provides a model rotation curve fit to both
the \halpha\ and stellar data, each parametrized as a hyperbolic tangent
function: $v_{\rm rot} = v_{\rm flat} \tanh(R/h_v)$.  However, our
primary result is based on the error-weighted mean of the deprojected
rotation-curve measurements for each dynamical tracer.  We only include
measurements within $\pm$30$\arcdeg$ of the major axis and construct
radial bins that are $2\farcs5$ wide and centered at 0.25, 0.5, 0.75,
1.0, and 1.25 $R_{\rm eff}$ --- $R_{\rm eff}$ is the effective radius
using the elliptical Petrosian analysis from the NASA-Sloan Atlas.
\comment{Check if the $R_{\rm eff}$ does or does not include the
multiplicative offset to match these Petrosian and Sersic $R_{\rm eff}$
in the mean.}  Specifically, we calculate the error-weighted mean and
standard deviation of $v_j = V_j/\cos\theta_j/\sin i$ and
%
\begin{equation}
%
\sigma_{a,j}^2 = (V_{{\rm H}\alpha,j}^2 - V_{\ast,j}^2) (\cos\theta_j
\sin i)^{-2},
%
\label{eq:dproj}
%
\end{equation}
%
where $V_j$ is the LOS measurement of each component in spaxel $j$
located at the in-plane polar coordinates $R_j$ and $\theta_j$ and $i$
is the disk inclination.

\comment{An example demonstrating the details of our measurements is
illustrated in Figure X.}

We calculate errors in $v_{{\rm H}\alpha}$ and $\sigma_a$ as the
quadrature sum of the error-weighted standard error (i.e., the
error-weighted standard deviation divided by $\sqrt{N}$) and the
propagated error in the error-weighted mean. \comment{Should we revisit
this?  Details of error calculation not all that important.  We're
dominated by intrinsic deviations in the regressions below.}

\section{Results}
\label{sec:results}

%%%%%%%%%%%%%%%%%%%%%%%%%%%%%%%%%%%%%%%%%%%%%%%%%%%%%%%%%%%%%%%%%%%%%%%%
\begin{figure}
%
\begin{center}
%
\includegraphics[width=1.0\columnwidth]{figs/mi_models.pdf}
%
\end{center}
%
\caption{
%
Global photometry and kinematic measurements at $R=R_{\rm eff}/2$ for
the AD sample: $M_i$ versus (a) $N-r$ color with each point colored
according to $\sigma_a^2$, (b) $v_{{\rm H}\alpha}$ with each point
colored according to $\sigma_a^2$, (c) $\sigma_a^2$ with each point
colored according to $v_{{\rm H}\alpha}$, and (d) $\sigma_a/v_{{\rm
H}\alpha}$ with each point colored by the $N-r$ color.  Panels (b), (c),
and (d) include the Spearman rank correlation coefficient, $r_s$, and
the linear regressions (solid lines) constructed from the parameters
provided in Table \ref{tab:lines}.  The dashed lines are offset from
linear regression by the modeled intrinsic scatter in the relation
($\varepsilon_y$).
%
}
%
\label{fig:correlation}
%
\end{figure}
%%%%%%%%%%%%%%%%%%%%%%%%%%%%%%%%%%%%%%%%%%%%%%%%%%%%%%%%%%%%%%%%%%%%%%%%

Figure \ref{fig:correlation}(a) shows the $(M_i, N-r)$ color-magnitude
diagram (see right column of Figure \ref{fig:sample}) with the points
colored according to $\sigma_a$ measured at $R=0.5R_{\rm eff}$.  There
is a clear trend where brighter galaxies have larger $\sigma_a$; the
remainder of the Figure examines this trend.

We plot $M_i$ versus $v_{{\rm H}\alpha}$, $\sigma_a^2$, and
$\sigma_a/v_{{\rm H}\alpha}$ in Figures \ref{fig:correlation}(b),
\ref{fig:correlation}(c), and \ref{fig:correlation}(d), respectively.
The points are colored according to the label to the right of each color
bar.  Each panel provides the Spearman rank-correlation coefficient,
$r_s$, of the plotted data with errors derived from $10^3$ bootstrap
simulations.  We have also used a Markov Chain Monte Carlo to sample the
Bayesian posterior distribution for a linear regression to the data in
each panel, incorporating the errors in both axes \citep[see,
e.g.][]{2010arXiv1008.4686H}; the errors in the kinematic quantities
always dominate over the absolute magnitude errors.  The fitted model is
a line in parametric form with an intrinsic Gaussian scatter
perpendicular to the line.  That is, the line is defined as
$\mathbf{l}(t) = \mathbf{l}_0 + t\ \hat{\mathbf{l}}$ for a generalized
coordinate $t$ along the line, an origin $\mathbf{l}_0 = \{x_0, y_0\}$,
and the unit vector $\hat{\mathbf{l}} = \{\cos\phi, \sin\phi\}$.  The
fitted parameters are $y_0$, $\phi$, and the dispersion of the intrinsic
Gaussian scatter about the line, $\varepsilon$; $x_0$ is fixed at the
median abscissa of the data being fitted ($M_{i,0} = -20.6$).  Uniform
priors are used for $y_0$ and $\phi$ and a logarithmic prior
\comment{check this is true} is used for $\varepsilon$
\citep{MacKay:itp}.  Using the returned samples of the posterior, we
also provide parameters for the slope-intercept form of the line --- $y
= mx + b$ where $m = \tan\phi$ and $b = y_0 - x_0 \tan\phi$ --- and the
scatter projected along the ordinate, $\varepsilon_y =
\varepsilon/|\cos\phi$|.  Table \ref{tab:lines} provides the median and
standard deviation of the marginalized distribution of each parameter;
these parameters have been used to construct the lines provided in
Figure \ref{fig:correlation}.

\comment{Show the PDFs?}

%%%%%%%%%%%%%%%%%%%%%%%%%%%%%%%%%%%%%%%%%%%%%%%%%%%%%%%%%%%%%%%%%%%%%%%%
\begin{deluxetable}{ c r r r }
\tabletypesize{\small}
\tablewidth{0pt}
\tablecaption{Linear Regressions for $M_i$ \comment{check m-b form}}
\tablehead{ & \multicolumn{3}{c}{Dependent Variable} \\ \cline{2-4} \\[-3pt]
 \colhead{Parameter} & \colhead{$\log(v_{{\rm H}\alpha})$} &
 \colhead{$\log(\sigma_a^2)$} &
 \colhead{$\log(\sigma_a/v_{{\rm H}\alpha})$} }
\startdata
          $y_0$ &      2.168 &       3.78 &     -0.269 \\
                & $\pm$0.003 &  $\pm$0.01 & $\pm$0.003 \\[2pt]
         $\phi$ &     170.70 &      160.1 &      -0.24 \\
                &  $\pm$0.15 &   $\pm$0.5 &  $\pm$0.17 \\[2pt]
  $\varepsilon$ &      0.069 &      0.209 &      0.068 \\
                & $\pm$0.001 & $\pm$0.005 & $\pm$0.002 \\[2pt] \hline \\[-4pt]
            $m$ &     -0.164 &     -0.362 &     -0.004 \\
                & $\pm$0.003 & $\pm$0.009 & $\pm$0.003 \\[2pt]
            $b$ &      -1.20 &      -3.69 &      -0.36 \\
                &  $\pm$0.05 &  $\pm$0.19 &  $\pm$0.06 \\[2pt]
$\varepsilon_y$ &      0.070 &      0.222 &      0.068 \\
                & $\pm$0.001 & $\pm$0.005 & $\pm$0.002
\enddata
\label{tab:lines}
\end{deluxetable}
%%%%%%%%%%%%%%%%%%%%%%%%%%%%%%%%%%%%%%%%%%%%%%%%%%%%%%%%%%%%%%%%%%%%%%%%

As expected, there is a strong correlation between $v_{{\rm H}\alpha}$
and $M_i$.  However, it is important to note that Figure
\ref{fig:correlation}(b) does not present the
\citet{1977A&A....54..661T} relation for our AD sample; the Figure gives
the rotation speed at $R=0.5R_{\rm eff}$, not a measure of the
full-width of the dynamically broadened line profile.  The slope of the
relation in Figure \ref{fig:correlation}(b) is steeper than a nominal
Tully-Fisher relation because galaxies at low luminosity tend to have
more slowly rising rotation curves \comment{refs}, which can be
confirmed by plotting $h_{v,{\rm H}\alpha}/R_{\rm eff}$ as a function of
$M_i$.  \comment{actually show this?}

Figure \ref{fig:correlation}(c) gives the direct representation of the
gradient in the point color seen in Figure \ref{fig:correlation}(a).
The data in this panel are highly correlated, both as determined by
$r_s$ and the fitted regression.  The intrinsic scatter increases from
0.07 dex (17\%) in $(M_i,v_{{\rm H}\alpha})$ to 0.21 dex (62\%) in
$(M_i,\sigma_a^2)$; however, this is close to the $\sim$0.1 dex scatter
if considering $(M_i,\sigma_a)$ instead.

If $\sigma_a$ or $v_{{\rm H}\alpha}$ was a fully orthogonal secondary
parameter in the distribution of $(M_i,v_{{\rm H}\alpha})$ or
$(M_i,\sigma_a^2)$, respectively, we should expect a gradient in the
point color in Figures \ref{fig:correlation}(b) and
\ref{fig:correlation}(c) {\em perpendicular} to the fitted regression.
However, we see that the primary gradient in point color is parallel to
the fitted regression.  This is consistent with the result that there is
little to no correlation between $M_i$ and $\sigma_a/v_{{\rm H}\alpha}$,
as shown in Figure \ref{fig:correlation}(d):  Both $r_s$ and $\phi$ are
small and only marginally significant with respect to their errors.
From equation \ref{eq:dproj}, this implies that the deprojected stellar
rotation is roughly a constant fraction of the ionized-gas rotation
speed, with $v_{\ast} \sim 0.9 v_{{\rm H}\alpha}$ \comment{give
scatter}.  Although the radius at which the AD was sampled is different,
this ratio is consistent with a similar ratio measured for
DiskMass-Survey galaxies \citep{2013A&A...557A.130M}.

\comment{Need to revisit this paragraph in the context of the
bulge-to-disk fraction.}  The points in Figure \ref{fig:correlation}(d)
are colored by the global $N-r$ color.  It is difficult to determine
from this illustration if their is any correlation between
$\sigma_a/v_{{\rm H}\alpha}$ and global galaxy color.  One might expect
such a correlation if, for example, galaxies with different global color
have different light-weightings of the thin- vs. thick-disk
\comment{refs; not sure there are relevant ones; Comer\'{o}n?}, which
propagates to a difference in which component dominates the dynamical
pressure in the disk midplane.  Figure \ref{fig:colortrend} assesses
this directly by plotting the error-weighted mean trend in
$\sigma_a/v_{{\rm H}\alpha}$ for quartiles of the $N-r$ distribution of
the AD sample.  The error bars represent the error-weighted standard
deviation of the data in each bin, whereas the error-weighted standard
error is always {\it smaller than the plotted point}.  This suggests a
marginal, yet statistically significant, detection of a slightly larger
$\sigma_a/v_{{\rm H}\alpha}$ in the reddest bin.

%; however, more data would be required to lend this result any
%statistical significance.

%%%%%%%%%%%%%%%%%%%%%%%%%%%%%%%%%%%%%%%%%%%%%%%%%%%%%%%%%%%%%%%%%%%%%%%%
\begin{figure}
%
\begin{center}
%
\includegraphics[width=1.0\columnwidth]{figs/adov_vs_color.pdf}
%
\end{center}
%
\caption{
%
The error-weighted mean trend in $\sigma_a/v_{{\rm H}\alpha}$ binned in
quartiles of the global $N-r$ color (black points).  The error bars
represent the error-weighted standard deviation of the data within each
bin; the error-weighted standard error is smaller than the size of each
black point.  The underlying gray histogram provides the number of
measurements in bins of $N-r$; each quartile contains $\sim$152
galaxies.
%
}
%
\label{fig:colortrend}
%
\end{figure}
%%%%%%%%%%%%%%%%%%%%%%%%%%%%%%%%%%%%%%%%%%%%%%%%%%%%%%%%%%%%%%%%%%%%%%%%

\section{Discussion}
\label{sec:discussion}

\subsection{Technical Concerns}

\comment{Sample bias}

\comment{Beam-smearing}

\subsection{Prospects}

\comment{If $\sigma_a$ is a direct proxy for $\sigma_R$ ...}

\bibliography{master}

\end{document}


% Under the nominal assumptions of (i) a radially independent rotation
% speed and SVE shape and (ii) a exponential dispersion profile with an
% e-folding length of twice the disk scale length ($h_R$, ref), equation
% \ref{eq:adformal} can be used to show asymmetric drift will be maximum
% at 1.25$h_R$, or approximately 0.7$R_{\rm eff}$.



%\begin{deluxetable*}{ r r r r r r r r }
%\tabletypesize{\small}
%\tablewidth{0pt}
%\tablecaption{Linear Regressions for $M_i$ [[Check slope-intercept form]]}
%\tablehead{ \colhead{Dependent} & \multicolumn{3}{c}{Parametric Form} && \multicolumn{3}{c}{Slope-intercept Form} \\
%    \cline{2-4} \cline{6-8}
%    \colhead{Variable} & \colhead{$y_0$} & \colhead{$\phi$} & \colhead{$\varepsilon$} 
%                        && \colhead{$m$} & \colhead{$b$} & \colhead{$\varepsilon_y$} }
%\startdata
% $v_{{\rm H}\alpha}$                &      2.168 &    170.70 &      0.069 &&     -0.164 &     -1.20 &      0.070 \\
%                                    & $\pm$0.003 & $\pm$0.15 & $\pm$0.001 && $\pm$0.003 & $\pm$0.05 & $\pm$0.001 \\[2pt]
% $\log(\sigma_a^2)$                 &       3.78 &     160.1 &      0.209 &&     -0.362 &     -3.69 &      0.222 \\
%                                    &  $\pm$0.01 &  $\pm$0.5 & $\pm$0.005 && $\pm$0.009 & $\pm$0.19 & $\pm$0.005 \\[2pt]
% $\log(\sigma_a/v_{{\rm H}\alpha})$ &     -0.269 &     -0.24 &      0.068 &&     -0.004 &     -0.36 &      0.068 \\
%                                    & $\pm$0.003 & $\pm$0.17 & $\pm$0.002 && $\pm$0.003 & $\pm$0.06 & $\pm$0.002
%\enddata
%\label{tab:lines}
%\end{deluxetable*}


